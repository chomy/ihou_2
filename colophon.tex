\chapter*{編集後記}
雑音製作所偽述部彙報 vol.2をお届けします。ってもう誰も読んでいないかな。

秋葉原にある電子部品やキットを販売されているaitendoさんが、DSPラジオキットを発売していることを聞いて、
思わず買ってしまいました。それを気に、以前から気になっていたDSPを使った復調について調べるいい機会になる
と思い、さっそく同人誌に解説記事を書いてみました。

IQ変調、復調は、現代の無線通信には書かせない変調、復調方式になっています。
次号はFMとDSPで何をやっているかの話を書く予定です。その次にQAMといったデジタル通信の話ができればいいな
と思っています。

最後に、この同人誌はすべてオープンソースソフトウェアを使って作成されました。Debian/GNU Linux、\TeX Live2012
psutils、git、Gummiなどなど。また日本語のフォントはIPAexフォントを埋め込んだPDFを印刷しています。
このような有益なソフトウェアを開発、維持、管理していただいているすべての皆様に感謝します。
また、このページまでたどり着いてくれた読者の方(おそらくあなただけです)に感謝します。
ありがとうございました。

\begin{flushright}
2013年7月 Keisuke Nakao (@jm6xxu) 
\end{flushright}

%\subsection*{参考文献}
%\begin{itemize}
%  \item 中島将光
%    「マイクロ波工学」 森北出版 ISBN4-627-71030-5
%\end{itemize}S
\clearpage
\mbox{}
\vspace{49em}\\
この作品はクリエイティブ・コモンズ・ライセンス 表示 - 継承 2.1 日本 の下に提供されています。このライセンスのコピーを見るためには、http://creativecommons.org/licenses/by-sa/2.1/jp/ をご覧ください。